%\chapter*{\markboth{致谢}{致谢}{致谢}}
\chapter*{\markboth{致谢}{致谢}{致\quad 谢}}
\addcontentsline{toc}{chapter}{致谢}

三年时间说短不短,说长不长,但是对我却是意义重大的三年,无论是学习还是生活,学校对我的影响都是终生难忘的。首先,我要感谢父母一如既往的默默支持,没有他们就没有我的今天,虽然远隔千山万里,也照顾不到我的学习和生活,但只要想到,不管我做怎样的决定,他们都会全力支持,我很感动;然后,我要感谢我的导师,从他那里我学到严谨治学的态度,他也给予了我最大的自由,这得以让我去一些技术公司实习,接触到最前沿的正在发生深刻变革的人工智能领域,这段实习经历除了让我开阔眼界,接触了深度学习技术和计算框架,更重要的是结识了老师木(一流科技CEO)和一些志同道合的同事,如深度学习算法研究者陈新鹏,计算框架开发者王笑舒等;此外,还要感谢新浪的总监高鹏,实习期间,除了基本业务外,让我做了很多我感兴趣的事,如学习 R 语言绘图系统和 R Markdown生态系统,最得益的莫过于见识了大数据平台的系统架构;最后我要感谢统计之都,特别是创始人谢益辉,除了使用他开发的工具打造毕业论文模板,使得论文排版工作量直接降低了几个量级,一年多以来,还一直对我的问题有问必答。三年来,帮助过我的老师,同学,同事,朋友太多,他们当中很多都直接或间接地帮助了我的毕业论文,人生最大的幸运莫过于结识你们。

\chapter*{\markboth{作者简介}{作者简介}{作者简介}}
\addcontentsline{toc}{chapter}{作者简介}

黄湘云,男(1992-),2015年毕业于中国矿业大学(北京),获理学学位;2018年毕业于中国矿业大学(北京),攻读硕士学位,专业为统计学,研究方向为数据分析与统计计算。

% 我是中国矿业大学大学(北京)理学院2015级的硕士研究生,师从李再兴教授,主修统计学专业,方向是数据分析与统计计算。跟随导师学习了线性模型等理论,在导师的帮助下,我选择空间广义线性混合效应模型及其应用作为毕设的题目,论文写作期间,学习了新的模型语言 Stan 和编程语言 Python,并且比较熟练地掌握了 R 语言,这对与模型实现和数据分析起了决定性作用。三年里,获得研究生二等奖学金两次,一等奖学金一次,研究生优秀学生称号一次。作为第一完成人,在统计之都发表三篇文章,现任统计之都成员,参与审稿等工作。

\begin{center}
{\kaishu \zihao{4} 在学期间参加科研项目}
\end{center}

\begin{enumerate}
\tightlist
\item 国家自然科学基金项目“混合模型的方差元素检验及函数型混合模型研究”项目组成员。项目编号:11671398。2017年01月-2020年12月
\end{enumerate}

\begin{center}
{\kaishu \zihao{4} 主要获奖}
\end{center}

\begin{enumerate}
\tightlist
\item 2015-2016年度获研究生优秀学生一等奖学金
\item 2016-2017年度获研究生优秀学生奖学金
\end{enumerate}


% \appendix % 附录开始
% \chapter{软件工具简介}
% 
% \section{Stan}
% 
% 呐,到这里朕的书差不多写完了,但还有几句话要交待,所以开个附录,再啰嗦几句,各位客官稍安勿躁、扶稳坐好。




\backmatter
