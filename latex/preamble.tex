%% 数学字体
\usepackage{amsfonts} % \mathbb \mathfrak
\usepackage{mathrsfs} % \mathscr
% included by pandoc-template
\usepackage{amssymb}  % for \varnothing
\usepackage{amsmath}
\usepackage{bm}

% 章节标题数字样式
\ctexset{
  chapter/name = {,},
  chapter/number = \arabic{chapter},
  chapter/numberformat = \sf, % 数字字体 Arial
  chapter/beforeskip = 16pt, % 标题前后的间距  12磅 小四号字体
  chapter/afterskip = 16pt,  % 
  chapter/fixskip = true, % 抑制多余的距离 
  chapter/format += \heiti\zihao{3},  % 一级标题
  section/numberformat = \rm,
  section/format += \heiti\zihao{4}\raggedright, % 二级标题
  section/beforeskip = 16pt, % 段前后的间距 同一级标题
  section/afterskip = 16pt,
  section/fixskip = true, 
  subsection/numberformat = \rm,
  subsection/format += \heiti\zihao{-4}\raggedright,  % 三级标题
  subsection/fixskip = true, 
  subsection/beforeskip = 16pt, % 段前后的间距 同一级标题
  subsection/afterskip = 16pt,
  % contentsname = {目\quad 录},
}

\renewcommand\appendix{\setcounter{secnumdepth}{-1}}

\usepackage{fancyhdr}
\pagestyle{fancy}
\fancyhf{}
% 设置文武线
\renewcommand{\headrule}{\hrule height1pt width\headwidth \vspace{3.0pt}\hrule width\headwidth}
% 设置左页页眉
\fancyhead[EC]{\kaishu 中国矿业大学~(北京) 硕士学位论文} % 左页也是奇数页 
% 设置右页页眉
\fancyhead[OC]{\kaishu \leftmark} % 右页也是偶数页
% 设置页脚
\fancyfoot[C]{\thepage} % 没有 E或O 则表示左页和右页一样的设置

% 专为 book 类设置新章节的首页
\fancypagestyle{plain}{ \fancyhf{} %
\fancyhead[EC]{\kaishu 中国矿业大学~(北京) 硕士学位论文}
\fancyhead[OC]{\kaishu \leftmark}
\fancyfoot[C]{\thepage}}

\graphicspath{{figures/}{_bookdown_files/}}
\frontmatter
