%% 数学字体
\usepackage{amsfonts} % \mathbb \mathfrak
\usepackage{mathrsfs} % \mathscr
% included by pandoc-template
\usepackage{amssymb}  % for \varnothing
\usepackage{amsmath}
\usepackage{bm}
% \usepackage{breqn} % provides automatic line breaking for big formulas
% \usepackage{mathtools, nccmath} % 
% \usepackage{nccmath}
% https://tex.stackexchange.com/questions/375170/vertical-spacing-of-equations-using-align-environment-following-short-texts?noredirect=1&lq=1
% 章节标题数字样式
\ctexset{
  chapter/name = {,},
  chapter/number = \arabic{chapter},
  chapter/numberformat = \sf, % 数字字体 Arial
  chapter/beforeskip = 16pt, % 标题前后的间距  即12磅
  chapter/afterskip = 16pt,  % 即6磅
  chapter/fixskip = true, % 标题与正文的距离
  chapter/format += \heiti\zihao{3},  % 一级标题
  section/numberformat = \rm,
  section/format += \heiti\zihao{4}\raggedright, % 二级标题
  section/beforeskip = 16pt, % 段前后的间距 同一级标题
  section/afterskip = 16pt,
  section/fixskip = true, % 标题与正文的距离  
  subsection/numberformat = \rm,
  subsection/format += \heiti\zihao{-4}\raggedright,  % 三级标题
  % contentsname = {目\quad 录},
}

%% 段首缩进40点 即两个汉字
\setlength\parindent{40pt}

\renewcommand\appendix{\setcounter{secnumdepth}{-1}}

\usepackage{fancyhdr}
\pagestyle{fancy}
\fancyhf{}
% 设置文武线
\renewcommand{\headrule}{\hrule height1pt width\headwidth \vspace{3.0pt}\hrule width\headwidth}
% 设置左页页眉
\fancyhead[EC]{\kaishu 中国矿业大学~(北京) 硕士学位论文} % 左页也是奇数页 
% 设置右页页眉
\fancyhead[OC]{\kaishu \leftmark} % 右页也是偶数页
% 设置页脚
\fancyfoot[C]{\thepage} % 没有 E或O 则表示左页和右页一样的设置

% 专为 book 类设置新章节的首页
\fancypagestyle{plain}{ \fancyhf{} %
\fancyhead[EC]{\kaishu 中国矿业大学~(北京) 硕士学位论文}
\fancyhead[OC]{\kaishu \leftmark}
\fancyfoot[C]{\thepage}}

\graphicspath{{figures/}{_bookdown_files/}}

% \usepackage{parskip}
% \setlength{\abovedisplayskip}{1pt} %3pt 个人觉得稍妥,可自行设置
% \setlength{\belowdisplayskip}{1pt}

\newenvironment{shrinkeq}[1]
{\bgroup
\addtolength\abovedisplayshortskip{#1}
\addtolength\abovedisplayskip{#1}
\addtolength\belowdisplayshortskip{#1}
\addtolength\belowdisplayskip{#1}}
{\egroup\ignorespacesafterend}

% \begin{shrinkeq}{-3ex}
% \end{shrinkeq}

% \usepackage{amsthm}
% \makeatletter
% \def\thm@space@setup{%
%   \thm@preskip=8pt plus 2pt minus 4pt
%   \thm@postskip=\thm@preskip
% }
% \makeatother
% 
% \newtheorem{theorem}{定理}
% \newtheorem{lemma}[theorem]{引理}

\frontmatter
